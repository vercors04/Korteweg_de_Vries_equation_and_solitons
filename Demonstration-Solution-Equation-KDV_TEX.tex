\documentclass[12pt,a4paper]{article}
\usepackage[utf8]{inputenc}
\usepackage[T1]{fontenc}
\usepackage{amsmath, amsfonts, amssymb}
\usepackage{geometry}
\usepackage{lmodern}

\geometry{hmargin=2cm,vmargin=1.5cm}
\setlength{\parindent}{0pt}
\setlength{\parskip}{5pt}

\begin{document}

%page 1 
\begin{center}
    \rule{\linewidth}{0.5pt} \\
    \Large \textbf{Equation de Korteweg-de Vries} \\
    \rule{\linewidth}{0.5pt} \\
    \vspace{0.5cm}
    \[ \frac{\partial u(x,t)}{\partial t}+\epsilon u(x,t)\frac{\partial u(x,t)}{\partial x}+\mu\frac{\partial^{3}u(x,t)}{\partial x^{3}}=0 \]
\end{center}


%page 2
On a (1) :
\[ \frac{\partial u(x,t)}{\partial t}+\epsilon u(x,t)\frac{\partial u(x,t)}{\partial x}+\mu\frac{\partial^{3}u(x,t)}{\partial x^{3}}=0 \]

(2) On pose $\xi=x-ct$
\[ u(x,t)=z(x-ct)=z(\xi) \]

Calcul des dérivées :
\[ \frac{\partial u}{\partial t}=\frac{dz}{dt}=\frac{dz}{d(x-ct)}\frac{\partial(x-ct)}{\partial t}=-c\frac{dz}{d\xi} \]
\[ \frac{\partial u}{\partial x}=\frac{dz}{dx}=\frac{dz}{d(x-ct)}\frac{\partial(x-ct)}{\partial x}=\frac{dz}{d\xi} \]
\[ \frac{\partial^{3}u}{\partial x^{3}}=\left(\frac{\partial}{\partial x}\right)^{3}u=\frac{d^{3}z}{d\xi^{3}} \]

On substitue (2) à (1) et on obtient :
\[ -c\frac{dz(\xi)}{d\xi}+\epsilon z(\xi)\frac{dz(\xi)}{d\xi}+\mu\frac{d^{3}z(\xi)}{d\xi^{3}}=0 \]

Transformation de la dérivée totale en dérivée partielle (intégration) :
\[ C_{1}=-c\int dz+\epsilon\int z dz+\mu\int\frac{d^{3}z(\xi)}{d^{3}\xi} d\xi \]
\[ C_{1}=-cz+\frac{\epsilon}{2}z^{2}+\mu\frac{d^{2}z(\xi)}{d\xi^{2}} \]


%page 3
On a donc :
\[ C_{1}=-cz+\mu\frac{d^{2}z}{d\xi^{2}}+\frac{\epsilon}{2}z^{2} \]

On multiplie par $\frac{dz}{d\xi}$ :
\[ C_{1}\frac{dz}{d\xi}=-cz\frac{dz}{d\xi}+\frac{\epsilon}{2}z^{2}\frac{dz}{d\xi}+\mu\frac{d^{2}z}{d\xi^{2}}\cdot\frac{dz}{d\xi} \]
\[ C_{1}dz=-cz dz+\frac{\epsilon}{2}z^{2}dz+\mu\frac{d^{2}z}{d\xi^{2}}dz \]

On intègre des deux côtés et on pose une constante d'intégration $C_2$ :
\[ C_{1}\int dz=-c\int z dz+\frac{\epsilon}{2}\int z^{2}dz+\mu\int\frac{d^{2}z}{d\xi^{2}}dz \]
\[ C_{1}z+C_{2}=-\frac{c}{2}z^{2}+\frac{\epsilon}{6}z^{3}+\frac{1}{2}\mu\left(\frac{dz}{d\xi}\right)^{2} \]

Lorsque $x \to \pm \infty$ on a $C_{1}=C_{2}=0$  \\
On a donc :
\[ 0=-\frac{c}{2}z^{2}+\frac{\epsilon}{6}z^{3}+\frac{1}{2}\mu\left(\frac{dz}{d\xi}\right)^{2} \]
\[ -\frac{1}{2}\mu\left(\frac{dz}{d\xi}\right)^{2}=-\frac{c}{2}z^{2}+\frac{\epsilon}{6}z^{3} \]
\[ \left(\frac{dz}{d\xi}\right)^{2}=\frac{c}{\mu}z^{2}-\frac{\epsilon}{3\mu}z^{3} \]
\[ \left(\frac{dz}{d\xi}\right)^{2}=z^{2}\left(\frac{c}{\mu}-\frac{\epsilon}{3\mu}z\right) \]



%page 4
On utilise la méthode de la séparation des variables :

\textbf{A finir ! Ou va le mu ??? jsp}

\end{document}